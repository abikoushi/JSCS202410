\documentclass[dvipdfmx, dvipsnames]{beamer}
\usetheme[secheader]{Boadilla}
% \usepackage{beamerthemesplit} // Activate for custom appearanced
%\setbeamertemplate{caption}[numbered]
\usefonttheme[onlymath]{serif} %数式をゴシックにしない
%\setbeamertemplate{blocks}[rounded] % Blockの影を消す
\useinnertheme{circles} % 箇条書きをシンプルに
\setbeamertemplate{navigation symbols}{} % ナビゲーションシンボルを消す
\setbeamertemplate{footline}[frame number] % フッターはスライド番号のみ
\setbeamercolor{page number in head/foot}{fg=black}
% \usepackage{beamerthemesplit} // Activate for custom appearance
%\setlength{\parindent}{1em}  %段落字下げ
\renewcommand{\figurename}{Fig}
\renewcommand{\tablename}{Tab}
\usepackage[caption=false]{subfig}
%\usepackage{blindtext}
\usepackage{tikz}
\usetikzlibrary{decorations.pathreplacing,calligraphy}
\usetikzlibrary{angles,quotes} % for pic
\setbeamerfont{itemize/enumerate subbody}{size=\normalsize} %subitem を同じ大きさに
%
%\setbeamertemplate{itemize subitem}{\normalsize\raise1.25pt\hbox{\donotcoloroutermaths$\blacktriangleright$}}  %to set the symbol size
\usetikzlibrary{shapes,positioning}
\usepackage{xcolor}

\def\mathunderline#1#2{\color{#1}\underline{{\color{black}#2}}\color{black}}

\setbeamercolor{block title}{bg=gray!10} %block title の背景
\setbeamercolor{block body}{bg=white} %block の中身

%list を black-right-triangle にするコマンド
\newcommand{\triangleitem}{
\setbeamertemplate{itemize item}{\color{RedOrange}$\blacktriangleright$}
\setbeamertemplate{itemize subitem}{\color{RedOrange}$\blacktriangleright$}
}

\title{疎行列の非負値行列因子分解のための\\効率的な近似推定法}
\date{2024年10月}

%\renewcommand*{\thefootnote}{\fnsymbol{footnote}}
%\setcounter{footnote}{0} 

\author {阿部興 \footnote{東京科学大学 総合研究院 難治疾患研究所} ・島村徹平$^1$}

\begin{document}
\frame{
\titlepage
}
\begin{frame}
\frametitle{動機}
\begin{figure}
\begin{tikzpicture}[every node/.style={text width=90pt, align=center}]
\node at (0,0)(curse){次元の呪い};
%\node[below = of curse](encurse){Curse of dimensionality};
\node[right = of curse] (bless){次元の祝い\footnote[frame]{Gelman, A. (2004).``The blessing of dimensionality'' \url{https://statmodeling.stat.columbia.edu/2004/10/27/the_blessing_of/}}};
%\node[below = of bless](enbless){Blessing of dimensionality};
\path[draw, very thick, ->](curse)--(bless);
\end{tikzpicture}
\end{figure}
\begin{columns}
\begin{column}{0.5\textwidth}
\begin{itemize}
\item[$\bigcirc$] 高次元の積分は難しい
\item[$\triangle$] 高次元のデータ分析はややこしい
\end{itemize}
\end{column}
\begin{column}{0.5\textwidth}
\begin{itemize}
\item[$\circledcirc$] 見えるもの(変数)が多いことは嬉しい!
 \color{gray}{たとえそれが ``0" だったとしても}
\end{itemize}
\end{column}
\end{columns}
\vspace{\baselineskip}
\centering 
疎(sparse)なデータ $\neq$ 汚いデータ
%\color{gray}{「0 vs. 0 で『有意差』を出したい」のような無理を言わなければ}
\end{frame}

\begin{frame}
\frametitle{非負値行列因子分解(NMF)の紹介}
\begin{itemize}
\item 得られたデータを低次元に射影して圧縮することでパターンを抽出
\item 非負性の制約により解釈がしやすい 
\end{itemize}
\end{frame}

\begin{frame}
\frametitle{疎行列とNMF}
\begin{itemize}
\item 観測のゼロ過剰や過分散をモデル化したケース\footnote{e.g. Abe, H., \& Yadohisa, H. (2017). A non-negative matrix factorization model based on the zero-inflated Tweedie distribution. Computational Statistics, 32(2), 475-499. \\ Gouvert, O., Oberlin, T., \& F\'evotte, C. (2020). Negative binomial matrix factorization. IEEE Signal Processing Letters, 27, 815-819.}
\item 分解で得られる行列を疎にしようとした議論\footnote{e.g.  Hyunsoo, K. \&  Haesun, P. (2007). Sparse non-negative matrix factorizations via alternating non-negativity-constrained least squares for microarray data analysis, Bioinformatics,  23 (12), 1495--1502. \\ Kim, H., \& Park, H. (2007). Sparse non-negative matrix factorizations via alternating non-negativity-constrained least squares for microarray data analysis. Bioinformatics, 23(12), 1495-1502.}
 \end{itemize}
 \end{frame}
\end{document} 